\documentclass[12pt,twocolumn,twoside]{conference}   %%
\usepackage[german, english]{babel}                  %%
%%% Vorgaben %%%%%%%%%%%%%%%%%%%%%%%%%%%%%%%%%%%%%%%%%%


\title{Technique and procedures of business analytics}

\author{Sean Payne}

\begin{document}
\twocolumn[
  \begin{@twocolumnfalse}
  \maketitle\thispagestyle{firststyle}
    \begin{abstract}
    \vspace{8pt}
    {\color{red} \textbf{NOCH MAL ÜBERARBEITEN} }
Das folgende Paper handelt von den Techniken und Vorgehensweisen von Business Analytics. Hier wird das Augenmerk auf das Verfahren der "Predictive Analytics" und seine verschiedenen Methoden gelegt und wie man es einsetzen kann, um historische Daten auszuwerten. Diese Auswertungen sollen Unternehmen in der Zukunft helfen, Entscheidungen treffen zu können.
    Frage: Was ist der Stand der Technik und der Wissenschaft bezüglich 
    
    \end{abstract}
    \vspace{16pt}
  \end{@twocolumnfalse}
]

\section{Introduction}
\begin{list}{•}
\item[*] 
\item[*] Der Forschungsstand bei Predictive Analytics
\item[•] Zeigen wo die Lücken im System sind
\item[•] Ziel dieses Papers
\item[•] Zweck dieses Papers
\item[•] Fragestellung meinerseits
\item[•] Hypothesen die ich aufstelle
\end{list}

\begin{list}{•}
\item[*] 
\item[*] Wie weit sind die predictive Analytics
\item[*] Was ist mit ihnen möglich

\item[*] Technischer Stand
	\begin{enumerate}
		\item Data Mining Systeme
		\item Predictive Analytic Systeme
	\end{enumerate}
	
\item[*] Wissenschaftlicher Stand
	\begin{enumerate}
		\item Methoden der Predictive Analtics
	\end{enumerate}
\end{list}


Bei Predictive Analytics, handelt es sich um ein Verfahren, welches historische Datensätze nutzt um somit 


\section{Methods}
\begin{list}{•}
\item Das Vorgehen bei der Identifikation der ausgewerteten Studien und die Bewertung ihrer Qualität.
\end{list}


\section{Results}
\begin{list}{•}
\item Zusammenfassen der Ergebnisse
\end{list}


\section{Discussion}
\begin{list}{•}
\item[•] Einordnen der Resultate in den Forschungsstand der Einleitung
\item[•] Auf bestehende Erkenntnisse und Hypothese Bezug nehmen
\item[•] Man startet mit den Ergebnissen und wendet diese dann auf die Allgemeinheit an
\end{list}


\subsection{Zitierbeispiel}
'The structure of the Definition Schema is a representation of the data model of SQL.' \cite{ISO9075-1:2011}


\newpage

\begin{thebibliography}{99}
	
	\bibitem{ISO9075-1:2011}
	ISO/IEC 9075-1: Information technology
	Database languages -SQL- ,Part 1: Framework
	(SQL/Framework), 4. Auflage, ISO Copyright Office,
	Genf 2011
	
	
\end{thebibliography}

\end{document}
