\documentclass[12pt,twocolumn,twoside]{conference}   %%
\usepackage[german, english]{babel}                  %%
%%% Vorgaben %%%%%%%%%%%%%%%%%%%%%%%%%%%%%%%%%%%%%%%%%%


\title{Technique and procedures of business analytics}

\author{Sean Payne}

\begin{document}
\twocolumn[
  \begin{@twocolumnfalse}
  \maketitle\thispagestyle{firststyle}
    \begin{abstract}
    \vspace{8pt}
    {\color{red} \textbf{NOCH MAL ÜBERARBEITEN} }
Das folgende Paper handelt von den Techniken und Vorgehensweisen von Business Analytics. Hier wird das Augenmerk auf das Verfahren der "Predictive Analytics" und seine verschiedenen Methoden gelegt und wie man es einsetzen kann, um historische Daten auszuwerten. Diese Auswertungen sollen Unternehmen in der Zukunft helfen, Entscheidungen treffen zu können.
    Frage: Was ist der Stand der Technik und der Wissenschaft bezüglich 
    
    \end{abstract}
    \vspace{16pt}
  \end{@twocolumnfalse}
]

\section{Introduction}


\begin{list}{}
	\item {\color{red} \textbf{Aufbau der Arbeit} }
	\begin{enumerate}
		\item{1.}  Business Analytics
			\begin{enumerate}
				\item Was sind Business Analytics?
				\item Wo werden sie eingesetzt?
				\item Welchen Nutzen haben Business Analytics
				\item Was macht Business Analytics aus?
			\end{enumerate}				
		\item Data Mining Systeme
		\begin{enumerate}
			\item Was ist Data Mining?
			\item Wo werden Data Mining Systeme angewendet?
			\item Welchen Nutzen habe ich von Data Mining Systeme
			\item Welche Arten von Data Mining Systemen gibt es?
			\item Was können Data Mining Systeme und was nicht?
		\end{enumerate}	
		
		\item Predictive Analytic Systeme
		\begin{enumerate}
			\item Was versteht man unter "Predictive Analytics"?
			\item Wo werden Predictive Analytics angewendet?
			\item Welchen Nutzen habe ich von Predictive Analytics?
			\item Welche Arten von Predictive Analytics gibt es?
			\item Wo liegen die Grenzen bei Predictive Analytics?
			\item Methoden der Predictive Analytics
		\end{enumerate}
	\end{enumerate}
\end{list}


\begin{list}{}
\item[*] Der Forschungsstand bei Business Analytics
\item[•] Zeigen wo die Lücken im System sind
\item[•] Ziel dieses Papers
\item[•] Zweck dieses Papers
\item[•] Fragestellung meinerseits
\item[•] Hypothesen die ich aufstelle
\end{list}





\section{Methods}
\begin{list}{•}
\item Das Vorgehen bei der Identifikation der ausgewerteten Studien und die Bewertung ihrer Qualität.
\end{list}


\section{Results}
\begin{list}{•}
\item Zusammenfassen der Ergebnisse
\end{list}


\section{Discussion}
\begin{list}{•}
\item[•] Einordnen der Resultate in den Forschungsstand der Einleitung
\item[•] Auf bestehende Erkenntnisse und Hypothese Bezug nehmen
\item[•] Man startet mit den Ergebnissen und wendet diese dann auf die Allgemeinheit an
\end{list}


\subsection{Zitierbeispiel}
'The structure of the Definition Schema is a representation of the data model of SQL.' \cite{ISO9075-1:2011}


\newpage

\begin{thebibliography}{99}
	
	\bibitem{ISO9075-1:2011}
	ISO/IEC 9075-1: Information technology
	Database languages -SQL- ,Part 1: Framework
	(SQL/Framework), 4. Auflage, ISO Copyright Office,
	Genf 2011
	
	
\end{thebibliography}

\end{document}
